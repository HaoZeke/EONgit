\documentclass{article}

\usepackage{hyperref}

\begin{document}
\title{eOn Documentation}
\maketitle
\section{Getting the Code}
The code is available to users with an account on theory.cm.utexas.edu. It can
be checked out of the subversion repository with the following command:
\begin{verbatim}
    svn checkout svn+ssh://username@theory.cm.utexas.edu/Groups/svn/eon 
\end{verbatim}
This will fetch a copy of the latest code to a local directory eon.
The work flow for making changes, after editing the files, is to first see what
files you have modified:
\begin{verbatim}
    svn status
\end{verbatim}
This will give you a one line per changed file output of what you have done
since you checked out the code. It is often a good idea to run:
\begin{verbatim}
    svn update
\end{verbatim}
next to make sure that no other developers have commited changes since you
checked out your copy of the code you should run:
\begin{verbatim}
    svn update
\end{verbatim}
to get the latest copy of the code. It is possible that some other developer
modified the same file in the same places that you have. This means that there
will be some conflicts to resolve. This is a rare thing to happen and you
should read the \href{http://svnbook.red-bean.com/en/1.5/index.html}{online documentation} to figure out how to handle it. Once you have seen what changes you have made with `svn status` and ensured that you have the latest version of the code with `svn update` you should commit your changes to the repository:
\begin{verbatim}
    svn commit -m "a brief message describing your changes"
\end{verbatim}

\section{Introduction}

\section{Input Files}
Three input files are required to run an akmc simulation. The config.ini file
which sets the options for the server code. The parameters.dat file which is
passed on to the client and the initial configuration of the chemical system to
be modeled.

\subsection{config.ini}

\subsection{parameters.dat}

\section{Output Files}

\end{document}

\documentclass{article}

\usepackage{hyperref}

\usepackage{amsmath}

\begin{document}
\title{eOn Install Documentation}
\author{Jonsson	 Group}
\maketitle
\section{Boinc client libraries on Ubuntu 10}

Take a look at http://boinc.berkeley.edu/trac/wiki/ServerIntro, which served as inspiration for this text.
Install all the packages listed under Cookbook for debian40:

1. libcurl3-dev install libcurl4-gnutls-dev instead

2. Build the boinc client libraries.
\begin{verbatim}
	./configure --disable-server
	make
	sudo make install
\end{verbatim}

\section{Boinc server setup on Ubuntu 10}

ake a look at http://boinc.berkeley.edu/trac/wiki/ServerIntro, which served as inspiration for this text.
Install all the packages listed under Cookbook for debian40:

1a. Make a new user(boincadm) to handle boinc:
\begin{verbatim}	
sudo useradd -m -s /bin/bash boincadm
\end{verbatim}

1b. Add www-data to group boinadm with:
\begin{verbatim}
sudo usermod -G boincadm www-data
\end{verbatim}

2a. Start mysql:
\begin{verbatim}
mysql -h localhost -u root [-p] 
\end{verbatim}

2b. Configure mysql
\begin{verbatim}
GRANT ALL ON *.* TO 'boincadm'@'localhost';
SET PASSWORD FOR 'boincadm'@'localhost'=''; //or any password you want
exit 
\end{verbatim}

3a. Get the latest stable code from BOINC
\begin{verbatim}
svn co http://boinc.berkeley.edu/svn/branches/server_stable
\end{verbatim}

3b. Rename the obtained folder to boinc
\begin{verbatim}
mv server_stable boinc
cd boinc
\end{verbatim}

3c. Build boinc server. Possible problem with autoreconf version. In \_autosetup edit the line 'if check\_version autoreconf 2.58' to 'if check\_version /usr/bin/autoreconf2.50'
\begin{verbatim}
./_autosetup   
./configure --disable-client
make 
\end{verbatim}


\section{Making a project}

Make sure the name of host and hostname is the same as your name to other computers. So /etc/hosts and /etc/hostname should specify that name.

1. From boinc/tools/ run:
\begin{verbatim}
./make_project eon 
\end{verbatim}

2. Follow the instructions in the readme file until the command bin/xadd.

3. Go to eon/html/ops and run:
\begin{verbatim}
\$ htpasswd -c .htpasswd <username>" this gives access to the ops site.
\end{verbatim}

4. Edit the function auth\_ops() in eon/html/project/project.ini by commenting out the default deny access and/or edit it to have some security.

5. Replace eon/project.xml and eon/templates with our version and copy the eon program to eon/program

6. Compile the client and place it in eon/apps/client/client\_version\_platform/client\_version\_platform

7. Add these daemons to eon/config.xml:
\begin{verbatim}
<daemon>
  <cmd>
    sample_trivial_validator -d 3 -app client
  </cmd>
</daemon>
<daemon>
  <cmd>
    sample_assimilator -d 3 -app client
  </cmd>
</daemon> 
\end{verbatim}

8. Run:
\begin{verbatim}
$ bin/xadd
$ bin/update\_versions
$ bin/start
\end{verbatim}

9. Edit paths and boinc paths in default\_config.ini and config.ini

\end{document}

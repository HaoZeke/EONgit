\documentclass{article}

\usepackage{hyperref}

\usepackage{amsmath}

\begin{document}
\title{eOn Install Documentation}
\author{Jonsson Group}
\maketitle

\section{Package manager}
The different components of a Boinc installation might be obtained using a package manager.  However, we have not tried such an installation.

\section{Boinc client libraries on Ubuntu 10}
Take a look at http://boinc.berkeley.edu/trac/wiki/ServerIntro, which served as inspiration for this text. In this document \texttt{...} is used to represent a path that depends on the specific installation.
%
\subsection{Prerequisites}
Start by ensuring that all the packages listed below are installed:
\begin{itemize}
\item libssl-dev
\item libglut3-dev
\item glutg3-dev
\item libglui-dev
\item libglitz-glx1-dev
\item libsdl1.2-dev
\item libcurl4-gnutls-dev
\item freeglut3
\item freeglut3-dev
\item libsm-dev
\item libice-dev
\item libxmu-dev
\item libxi-dev
\item libx11-dev
\item libjpeg62-dev
\item libgtk2.0-0
\item libgtk2.0-0-dev
\end{itemize}
%
\subsection{Making the Boinc client libraries}
%Build the boinc client libraries.
\begin{verbatim}
	./configure --disable-server
	make
	sudo make install
\end{verbatim}
Where the install option makes the created libraries and header files accessible system wide.
%
\section{Boinc server setup on Ubuntu 10}
Take a look at http://boinc.berkeley.edu/trac/wiki/ServerIntro, which served as inspiration for this text.
%
\subsection{Prerequisites}
Start by ensuring that all the packages listed below are installed:
\begin{itemize}
\item apache2-mpm-prefork
\item libapache2-mod-php5
\item mysql-client-5.0
\item mysql-server-5.0
\item php5-mysql
\item php5-cli
\item php5-gd
\item phpmyadmin
\item python-mysqldb
\item libmysql++-dev
\item libssl-dev
\end{itemize}
%
\subsubsection{boincadm user}
Make a new user(boincadm) to handle boinc:
\begin{verbatim}
sudo useradd -m -s /bin/bash boincadm
\end{verbatim}
%
Add www-data to group boincadm with:
\begin{verbatim}
sudo usermod -G boincadm www-data
\end{verbatim}
%
\subsubsection{MySQL}
Start mysql:
\begin{verbatim}
mysql -h localhost -u root [-p]
\end{verbatim}
Where the \texttt{-p} only should be used if the MySQL \texttt{root} user got a password.
%
Configure mysql
\begin{verbatim}
GRANT ALL ON *.* TO 'boincadm'@'localhost';
SET PASSWORD FOR 'boincadm'@'localhost'='XXX';
exit
\end{verbatim}
%
where \texttt{XXX} represents the password.
%
\subsection{Making the Boinc server software using boincadm}
Start by changing user to boincadm:
\begin{verbatim}
su boincadm
\end{verbatim}
It might be nesseary to use \texttt{sudo}.

Get the latest stable code from BOINC
\begin{verbatim}
svn co http://boinc.berkeley.edu/svn/branches/server_stable
\end{verbatim}
%
Rename the obtained folder to boinc
\begin{verbatim}
mv server_stable boinc
cd boinc
\end{verbatim}
%
Build boinc server.
\begin{verbatim}
./_autosetup
./configure --disable-client
make
\end{verbatim}
%
If \texttt{./autosetup} complains about a too old version of autoreconf the change line \texttt{if check\_version autoreconf 2.58} to \texttt{if check\_version /usr/bin/autoreconf2.50} in this file.
%
\section{Making a project}
Make sure the hostname is the same as the address used when connecting to the computer. This setting are specified in \texttt{/etc/hosts} and \texttt{/etc/hostname}.

Enter the Boinc tools directiory and create a new project:
\begin{verbatim}
cd .../boinc/tools
./make_project eon
\end{verbatim}
where \texttt{eon} can be replaced with another project name if preferred.
Follow the instructions in the created readme file until the command bin/xadd.

\subsection{Secure the ops page}
Go to eon/html/ops and run:
\begin{verbatim}
cd .../eon/html/ops
htpasswd -c .htpasswd USERNAME
\end{verbatim}
where \texttt{USERNAME} is the name of the user that should have access to the \texttt{ops} page.
In \texttt{eon/html/project/project.inc} edit \texttt{function auth\_ops()} by changing the default deny access \texttt{auth\_ops\_deny()} to \texttt{//auth\_ops\_deny()}.
\begin{verbatim}
cd .../eon/html/project/
nano  project.inc
\end{verbatim}

\subsection{Prepare the eOn to use the Boinc Communicator}
Go to the root of the project and replace \texttt{project.xml} with the one provided by the eon source code in the directory boinc
\begin{verbatim}
cd .../eon
mv project.xml project.xml.org
cp /PATH_EON_SOURCE_CODE/boinc/project.xml .
\end{verbatim}
%
Furthermore copy the template files:
\begin{verbatim}
cp /PATH_EON_SOURCE_CODE/boinc/client_re.xml ./template/client_re.xml
cp /PATH_EON_SOURCE_CODE/boinc/client_wu.xml ./template/client_wu.xml
\end{verbatim}
%
Compile the eon client with Boinc support but without fortran potentials:
\begin{verbatim}
cd /PATH_EON_SOURCE_CODE/client
make BOINC=1 NO_FORTRAN=1
\end{verbatim}
%
Copy the obtained executeable to the application area of the Boinc project
\begin{verbatim}
cd .../eon/apps/
mkdir client
cd client
mkdir client_VERSION_PLATFORM
cd client_VERSION_PLATFORM
cp /PATH_EON_SOURCE_CODE/client client_VERSION_PLATFORM
\end{verbatim}
where the \texttt{client\_VERSION\_PLATFORM} would be \texttt{client\_1.0\_i686-pc-linux-gnu} for the first version (\texttt{1.0}) of the executeable running on intel architecture on a linux platform (\texttt{i686-pc-linux-gnu}).
Add the following deamons to the \texttt{.../eon/config.xml}:
\begin{verbatim}
<daemon>
  <cmd>
    sample_trivial_validator -d 3 -app client
  </cmd>
</daemon>
<daemon>
  <cmd>
    sample_assimilator -d 3 -app client
  </cmd>
</daemon>
\end{verbatim}
%
To get the newly added application into the Boinc go to the root and run:
\begin{verbatim}
cd .../eon
./bin/xadd
./bin/update_versions
./bin/start
\end{verbatim}
%
Add the environmental variable \texttt{BOINC\_PROJECT\_DIR} containing the path to the project:
\begin{verbatim}
cd
nano .bash_profile
BOINC_PROJECT_DIR=".../eon"
export BOINC_PROJECT_DIR
\end{verbatim}
and finally change the paths stated in the eon configuration file \texttt{config.ini} among the eon server files.

\end{document}
